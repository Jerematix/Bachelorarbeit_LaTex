\documentclass[a4paper, oneside, 12pt]{article}

\usepackage[utf8]{inputenc}
\usepackage[T1]{fontenc}
\usepackage{graphicx}
\usepackage{longtable}
\usepackage{caption}
\usepackage{titling}
\usepackage{amssymb}
\usepackage{textcomp}
\usepackage{booktabs}
\usepackage{pdflscape}
\usepackage{apacite}
\usepackage{adjustbox}
\usepackage{float}
\usepackage{hyperref}
\usepackage{rotating}
\hypersetup{
    colorlinks,
    citecolor=black,
    filecolor=black,
    linkcolor=black,
    urlcolor=blue
}
\renewcommand\maketitlehooka{\null\mbox{}\vfill}
\renewcommand\maketitlehookd{\vfill\null}

% set margins for double-sided printing
\usepackage[left=2.5cm, right=2.5cm, top=2.5cm, bottom=2.5cm, bindingoffset=1.5cm, head=15pt]{geometry}
\usepackage{setspace}
\onehalfspacing
% set headers
\usepackage{fancyhdr}
\pagestyle{fancy}
\fancyhead{}
\fancyfoot{}
\fancyhead[L,RO]{\textsl{\leftmark}}
\fancyhead[R,LO]{\thesisauthor}
\fancyfoot[C]{\thepage}
\renewcommand{\headrulewidth}{0.4pt}
\renewcommand{\footrulewidth}{0pt}

\usepackage[backend=biblatex, giveninits=true, style=apa, maxcitenames=2, ibidtracker=false]{biblatex}
\usepackage{amsmath}
\addbibresource{main.bib}
\pagenumbering{gobble}

\title{Analyzing the Effects of Public Transit Schedules on Shared Mobility Usage}

\newcommand{\thesisdate}{July 26, 2022}
\newcommand{\thesisauthor}{Jeremy Meidinger}
\newcommand{\studentID}{7364610}
\newcommand{\thesistype}{Bachelor Thesis} % Set either to Bachelor or Master
\newcommand{\supervisor}{Univ.-Prof. Dr. Wolfgang Ketter}
\newcommand{\cosupervisor}{Janik Muires}



\begin{document}

    %%%%%%%%%%%%%%%%%%%%%%%%%%%%%%%%%%%%%%%%%%%%%%%%%%%%%%%%%%%%%
%TITLE PAGE
%%%%%%%%%%%%%%%%%%%%%%%%%%%%%%%%%%%%%%%%%%%%%%%%%%%%%%%%%%%%%
    \makeatletter
    \begin{titlepage}
        \begin{center}
            \vspace*{1cm}

            \Large
            \textbf{\@title}

            \vspace{1.5cm}

            \thesistype{}

            \vspace{1cm}

            \begin{figure}[htbp]
                \centering
                \includegraphics[width=.5\linewidth]{./Figures/UoC_Logo}
            \end{figure}

            \vspace{1cm}

            \large
            \textbf{Author}: \thesisauthor{} (Student ID: \studentID{})\\
            \large
            \textbf{Supervisor}: \supervisor{}\\
            \large
            \textbf{Co-Supervisor}: \cosupervisor{}

            \vspace{1cm}
            \large
            Department of Information Systems for Sustainable Society\\
            Faculty of Management, Economics and Social Sciences\\
            University of Cologne\\

            \vspace{1cm}
            \thesisdate

        \end{center}
    \end{titlepage}
    \makeatother

    \clearpage
    \thispagestyle{empty}
    \section*{Eidesstattliche Versicherung}
    \label{sec:SOOA}

    \vspace{2.5cm}

    Hiermit versichere ich an Eides statt, dass ich die vorliegende Arbeit selbstständig und ohne die Benutzung anderer als der angegebenen Hilfsmittel angefertigt habe. Alle Stellen, die wörtlich oder sinngemäß aus veröffentlichten und nicht veröffentlichten Schriften entnommen wurden, sind als solche kenntlich gemacht. Die Arbeit ist in gleicher oder ähnlicher Form oder auszugsweise im Rahmen einer anderen Prüfung noch nicht vorgelegt worden. Ich versichere, dass die eingereichte elektronische Fassung der eingereichten Druckfassung vollständig entspricht.

    \vspace{1cm}

    \noindent
    Die Strafbarkeit einer falschen eidesstattlichen Versicherung ist mir bekannt, namentlich die Strafandrohung gemäß § 156 StGB bis zu drei Jahren Freiheitsstrafe oder Geldstrafe bei vorsätzlicher Begehung der Tat bzw. gemäß § 161 Abs. 1 StGB bis zu einem Jahr Freiheitsstrafe oder Geldstrafe bei fahrlässiger Begehung.

    \vspace{3cm}
    \noindent
    \textbf{\thesisauthor{}}

    \vspace{0.5cm}
    \noindent
    Köln, den 26.07.2022

    %%%%%%%%%%%%%%%%%%%%%%%%%%%%%%%%%%%%%%%%%%%%%%%%%%%%%%%%%%%%%
    %TOC,TOF,TOT
    %%%%%%%%%%%%%%%%%%%%%%%%%%%%%%%%%%%%%%%%%%%%%%%%%%%%%%%%%%%%%
    \clearpage
    \pagenumbering{Roman}
    \tableofcontents
    \clearpage
    \listoffigures
    \clearpage
    \listoftables
    \clearpage

    \pagenumbering{arabic}


    \section{Introduction}

    \subsection{Problem Statement}

    Sustainability concerns are getting much more important in today's times as effects like climate change and pollution are advancing and risking of making the earth an uninhabitable space.
    According to the \cite{EEA2021} the transportation sector is one of the biggest contributor to greenhouse gas emissions and thus climate change.
    Urban mobility is one of the leading contributing factors to pollution and greenhouse gas emission (\cite{EEA2021}).
    The rise of the car in many regions has made the cities more uninhabitable through pollution and transit worse due to congestion (\cite{Fan2019}, p.1).

    To solve these challenges that have to be faced, society has to look, at alternative ways to car centric transportation.
    This means encouraging inhabitants of cities to go multimodal (use multiple types of transport in general) and intermodal (use multiple types of transport in one trip).
    One way to encourage intermodality, is encouraging public transit, since it is one alternative, which has already been established for a while.
    It is an efficient way to transport many people in a short amount of time, but because of its shortcomings it is not very well liked (\cite{Fan2019}).
    One of these shortcomings is that it is not always possible to use public transit, because of its schedule, or delays and cancellations, that delay transportation that would have otherwise happened.
    If one wants to travel from point A to point B, they have to wait for the public transit to arrive, and thus increase their commute time, without a benefit for themselves.
    The second major shortcoming is that, there is the first and last mile problem, riders of public transit have to pass the distance from their point to public transit (access) and from the public transit to their destination (egress) (\cite{Boecker2020}, p.390).
    Another major shortcoming is, that the service level is perceived to be lower, due to the riders not having their own personal space, but that they are travelling together with other riders in a single vehicle (\cite{Fan2019}, p.12).
    These problems are, that is an extensive issues for public transit, and public transit providers have to work on to improve  (\cite{Fan2019}, p.1-2).

    A solution to the first and last mile problem is to use a shared mobility system, which is a shared economy platform which is enabled by IT (\cite{Puschmann2016}, p.93), that allows people to rent modes of transport like bikes, cars and scooters for a short amount of time, for a typically relatively low fee.
    These means of transport are "publicly" available to all people, if they are willing to pay the ride fares and charges the providers demand.
    Typically these services operate within a confined service area, that the user has to start and end in, these service areas are mostly confined to the densest areas of the city.
    Altough there are these service area models, for bikes there are also older, dock based models, that only allow the user to start and stop at one of these docks (\cite{Boecker2020}, p.392).
    This has the advantage that as long as there is a shared mobility system, people can travel freely, without having to worry about the public transit schedules or going long distances by foot.

    Shared mobility has a couple different possible modes of transportation, some of them are: bike, car, kick scooter, and scooters.
    Though there has been research linking public transit to shared mobility (e.g \cite{Guo2020}, \cite{Lin2019}, \cite{Grosshuesch2020}, \cite{Fan2019}), there is still a lot of research missing, most of it is focused on the first and last mile of public transit.
    On top of that, most of this research is international, mostly focused on China and the US and not about the German transportation systems.
    Germany is known for regions like the Rheinland, or Ruhrgebiet, where there is high density population and with very much economic power, because these regions are so dense, intercity/regional transit is also something that has to be looked at.
    Added to that, there has been not much research on the effects of public transit schedules, meaning the density of public transit on shared mobility usage.
    This research could help public transit providers to better understand the interconnections their service has with shared mobility and thus improve their schedules and efficiency.
    Furthermore, it could help shared mobility providers to learn more about the interconnection of public transit and shared mobility, which could lead to better revenue planning and a better market and customer understanding.

    In this research I proceed as follows:
    I begin by reviewing prior literature on the interconnections of shared mobility and public transit, then I go into detail about the ongoing development of intermodality and multimodality in the context of urban mobility and I finish the literature review by explaining the importance of information systems for shared mobility.
    I then set up further hypotheses that break down the main research question, look at the empirical data and set up a model to validate these hypotheses.
    Lastly, I analyze the empirical results and conclude by discussing managerial implications, research implications, limitations, and contributions of my research.

    \subsection{Research Objective}

    The research objective is to find out if there is any link between public transit schedules and shared mobility usage, especially if a more or less frequent public transit schedule leads to a higher usage of shared mobility.
    To evaluate that, I got historical data from the public transit schedules and shared mobility usage from months in 2019-2020.
    The research question could be phrased as "Will a difference in public transit schedules lead to an increase/decrease in shared mobility usage and do the different modes react differently to differences in public transit and if so how?".
    To reach the to answer the question, the research will be divided it in the following substeps:

    \begin{enumerate}
        \item In addition to the provided data from the chair, search for more data, that helps to answer the question.
        \item Get the data in shape to use for analysis.
        \item Analyze the data and try to find influencing factors in the public transit schedules that lead to higher/lower shared mobility usage.
        \item Set up models that can be used to predict the usage of shared mobility based on the public transit schedules.
        \item Test the models and see if it can answer the research question and compare the models to one another
    \end{enumerate}

    To answer the research question according to the established substeps and the hypotheses models have to be developed.
    I developed a total of 12 models all following the following structure of dependent and independent variables.
    Where they differ is by examining different modes of transport, there are 3 in total: car, kick scooter and bike and of course putting all together.
    Then I will look at whether it makes a difference for each of the different modes of transport by looking at the starts and stops seperately and in one model looking at them together.
    See table \ref{tab:models} to get an overview of the 12 models we will have a look at in the following text.

    With these substeps I will be able to answer the initial research question.


    \section{Theoretical Background}

    \subsection{The Interconnections between Public Transit and Shared Mobility Usage}

    The most common connection of public transit and shared mobility is mostly stated as first and last mile travel, meaning the distance one has to overcome from the start to public transport and from public transit to the destination respecitvely (\cite{Grosshuesch2020}).
    To understand how public transport and shared mobility are interconnected, you first have to understand shared mobility, generally there are a couple of possible alternatives when it comes to shared mobility.
    Firstly there are bikes and e scooters, which are mostly the same except for bikes being used for slightly longer trips on average (\cite{Schellong2019}, p.3).
    They are mostly used for very short trips, for last mile applications, and bikes are generally considered to be a greener option although this is a disputed question.
    \cite{Hollingsworth2019} showed using monte carlo analyses that the chance for e-scooters being more eco friendly than dockless bikes is two thirds (\cite{Hollingsworth2019}, p.8), due to the high displacement costs of dockless bikes, which goes against the general notion, that they are more sustainable.
    But it is important to note, that eco friendliness not only consists of the $co ^ 2$ emissions but also of the amount of waste generated, which is generally more for e scooters due to their short lifespan and more advanced manufacturing processes including lithium (\cite{Hollingsworth2019}, p.4).
    Even if it is not very clear if e scooters or dockless bikes are more sustainable, it is generally assumed, that these sharing systems are more environmentally friendly, than passenger cars (\cite{Severengiz2020}, p.6, \cite{Hollingsworth2019}, p.8).
    Coming to the advantage that it is more environmentally friendly seen by itself, it is often coupled with tram and bus usage, which makes it even more environmentally friendly if the trip is seen as a whole, depending on the occupancy of the public transit (\cite{Hollingsworth2019}, p.8).
    There are generally also the options of having the vehicles free floating or having stations for these vehicles, which is most often for bikes, because it alleviates some of the issues for the providers and makes their access easier depending on the distribution fo the stations.
    But of course stations for these shared mobility systems also have a similar first and last mile issue, because some rest distance to the destination still remains most of the time (\cite{Grosshuesch2020}, p.851).
    Some cities especially in the US, try to campaign against these free floating vehicles, but this hampers innovation and makes it more difficult to solve the first and last mile issue and ultimately harm commuters (\cite{Grosshuesch2020}, p.867).
    One could also say now, that cities can fix the problem of too much traffic just by building more roads, unfortunately that is something which would only intuitively help (\cite{Lee1999}, p.68).
    This is because more road building leads to the phenomenon of induced demand, it can be quickly described when looking at travel as being a cost function, that's mostly dependent on time, if by more road building the cost to travel gets cheaper, more people will drive (\cite{Lee1999}).
    This goes on as long as there is an equilibrium again, where the mean traffic makes the cost so expensive, that switching to the car makes no sense anymore, but by then a larger share of the population uses cars to get around.
    But this is excactly not, what cities should encourage, but there is also research that leads to believe, that induced demand can also work for public rail networks, so cities should focusing on encouraging this travel instead of building roads (\cite{Givoni2013}, p.737).

    Furthermore, alternating investments from public transit to accomodating shared mobility, could alleviate some of the cities mobility struggles (\cite{Grosshuesch2020}, p.849).
    It due to ride hailing services relying on cars, they are not an end solution to the problem of urban transportation, but should be more used to accomodate public transit (\cite{Grosshuesch2020}, p.851).
    Cities should focus more on making public transit better and solving the first and last mile through e scooters and bicycles.

    In a paper by \cite{Boecker2020}, they collected data consisting of 2 years ride data and also data about the riders themselves from the Norwegian city of Oslo.
    They found, that bike rental frequencies are much higher to get to public transport and from public transport (\cite{Boecker2020}, p.398).
    This connection of bikes to get to public transportation and bikes to get from the public transportation stations to the final destination, these are typical first and last mile usages.
    Furthermore they found out, that mostly younger people and males use bikes as a connection to public transit stations, especially middle aged people use them much less frequently.
    They also find out, that riders living in inner Oslo more often use the bikes to get from start to finish, while outer Oslo residents, use them more frequently to get to public transit stations.
    They conclude, that shared mobility has the potential to enhance public transit, with differences between start and stop, bikes especially make sense for egress (last mile) trips, perpendicular to the metro lines (\cite{Boecker2020}, p.399).

    In another study by \cite{Fan2019} they looked at how the introduction of bike sharing services in china impacts the urban mobility choices of people.
    They found out, that around 5,5\% of automobile travelers changed to public transport after the introduction of bike sharing services.
    This relatively low number is explained with the relatively low service level of public transportation services and the aspirations of many chinese people to one day own a car, as 42\% of chinese young people have no car available, but 91\% plan to purchase one within the next 10 years (\cite{Fan2019}, p.12).
    This effect could also be explained, by the earlier discussed effect of displaying wealth and social status through car ownership, although this trend is going back in the western countries (\cite{Canzler2016}, p.10), as sustainability concerns rise, it seems to generally grow in China, due to Chinas big economical growth (\cite{Zhu2012}, p.317).
    To break these trends and to enhance the interconnections between public transit and shared mobility it is important, that public service providers improve their service level.

    To recap this section: Shared mobility and public transit are very sustainable, many young people are already willing to use shared mobility in tandem with public transit, but it is important to encourage other age groups to use shared mobility more often.
    Also older people should be encouraged to use public transit more often and leave the car as their status symbol, cities also have to support shared mobility especially the free floating kind, because it moderates positively on public transit usage.

    \subsection{Ongoing development of intermodality and multimodality in the context of urban mobility}

    % Transformation: Von Groß nach klein gehen, groß einführen und immer feiner werden mit der Zeit, am Ende kurzer Recap

    Urban mobility is changing with new introductions to the mobility landscape, like shared mobility platforms and ride hailing, these platforms indirectly encourage users to travel multimodally (Using multiple modes of transport in general) and intermodally (Using multiple modes of transport within a single trip).

    A big factor in the recent developments of multimodal traveling has been the smartphone, it allows travelers to get new information about travel on the go and in real time (\cite{Aguilera2019}, p.9).
    This allows users to change their travel plans on the go, and with higher precision than ever before, adapting to unpredictable scenarios like traffic jams, public transit delays etc (\cite{Aguilera2019}, p.9, \cite{Ben-Elia2015}, p.19)
    However this information is not enough on it's own to contribute to major change, research has shown that upon getting an application that allows people to get trip information multimodally does not necessarily improve their willingness to use multimodal transport (\cite{Aguilera2019}, p.11).
    The explanation stated for that is, that the advancements in technology and mobile technology also benefit the car transport, like being able to see trafic jams in and thus optimizing routes in real time (\cite{Aguilera2019}, p.11).
    These effects are so far not that well understood and have to be looked at more closely, but so far it looks like the smartphone makes existing travel more efficiently instead of changing the modals (\cite{Aguilera2019}, p.12).
    Although \cite{Aguilera2019} are not stating, that there is a complete overhaul in the way of thinking about traffic and it becomes much more spontaneos, and things like promotions in mobility apps, could lead to a change to more multimodal travel, this tactic has also been applied by many scooter providers with them offering very aggressive discounts (\cite{Schellong2019}, p.4)

    For others, the case is much clearer, the smartphone is even called "the digital key to an intermodal world." (\cite{Canzler2016}, p.8) and it revolutionizes travel completely and is the driving factor in the development of intermodal travel (\cite{Canzler2016}, p.8).
    This is explained by people being able to communicate online with their peers, meaning that the spatial component in communication gets removed (\cite{Canzler2016}, p.8).
    \cite{Canzler2016} also speak about a general change in the way cars are seen in general, especially in the past, cars were seen as a status symbol of wealth and being able to make individual travel (\cite{Canzler2016}, p.7).
    In the current zeitgeist and presumably more in the future, cars will be seen as a means to an end, and not as a status symbol anymore, and just as a for example through car sharing quick way to get from point A to point B (\cite{Canzler2016}, p.7).
    This is also partly due to car sharing services not putting emphasis on owning a car individually, but to the service of quick personal transportation, this is also similar for ride hailing services, where the car does not matter nearly as much, as the car getting there within 5 minutes.
    The car losing its value as status symbol and as \cite{Canzler2016} puts it "There is no going back." (\cite{Canzler2016}, p.10).

    \ \cite{Babar2017} investigated the effects of public transit on ride hailing services.
    They researched about which factors can increase or decrease the usage of ride hailing services like Uber and Lyft.
    To test that, they looked at 833 US cities, that introduced Uber's services and compared the public transit usage before and after.
    They developed a model influenced by the following factors: "Size of the local population, rates of violent crime, weather, gas prices, transit riders' average trip distance and the overall quality of public transit options." (\cite{Babar2017}, p.1).
    They found out that generally, the usage of buses decreases and the usage of trains increases, but this effect also varies widely with some other factors.
    Factors that generally lower bus usage like violent crimes, labor force, hot and snowy days lowered the effect of the introduction of ride hailing services, while the transit quality, the average trip length and gas prices positively moderated the effect ride hailing has on public transit (\cite{Babar2017}, p.28)
    To add to this point, it has to be stated, that of course ride-hailing is also not the best solution to the traffic problems of most cities, but since it can lead to intermodal transport with other public transportation, and enable some people who only occasionally need a car to completely refrain from buying one (\cite{Aguilera2019}, p.25).
    But of course ride hailing can also have the risk of people not changing their travel habits at all and thus making the problems of pollution and congestion only worse (\cite{Qin2018}).
    Mostly people who live in urban areas with good public transport options, opt to use these services, it makes travel for them cheaper having to maintain less cars or having no cars at all, because it helps them going without a car without completely changing their travel habits (\cite{Aguilera2019}, p.25).
    This contributes to households, who use these sharing mobility platforms, having a lower car use in general compared to similar households.
    This research shows, that urban mobility is evolving and well established means of mobility like public transit get influenced by new advancements like ride hailing services, partly increasing multimodal and intermodal travel, but also partly diminishing it.

    Furthermore, the introduction of free floating vehicle platforms can lower the number of the environmentally seen generally worse ride hailing services (\cite{Qin2018}).
    In a study in Chengdu, China it was found out, that the general pick up rate from bus stations decreased by 5\% for distances between 1 to 2 miles, but the usage at train stations decreased from 1 to 2 miles (\cite{Qin2018}, p.23).
    Also, the impacts of the introductions of second and third bike sharing platform were analyzed and the second still had significant impacts, but the 3rd introduction had mostly insignificant impacts (\cite{Qin2018}, p.25).
    This could imply that this effect has an upper limit, bound by the inherent willingness to use these sharing alternatives over the generally better travel experience of ride hailing.

    To understand how multimodality and intermodality develops you first have to understand how the status quo for multimodality and intermodality in urban mobility is.
    In a paper by \cite{Gebhardt2016}, they look at travel surveys from Berlin and Hamburg, Germany and they state that of the general population only around 17\% are travelling multimodally on a regular basis (\cite{Gebhardt2016}, p.1187).
    Also younger people tend to travel more multimodally, the age group 20\-29 being the strongest of all and with around 34\% in Berlin and 28\% in Hamburg.
    Also \cite{Gebhardt2016} state that the main reasons for people being multimodal is having no car available or having to arrange themselves with other members of the household (\cite{Gebhardt2016}, p.1187).
    This speaks to how multimodality is often seen, it is sort of seen as a necessary evil, that people mostly would not have chosen if they had a car easily and readily accessible.
    This peception needs to change if urban mobility wants to be accessible and feasible for every one, as it is simply not feasible to have single monomodal car travel for anyone, because of space constraints and congestion caused by the monomodal car approach (\cite{Gebhardt2016}).

    \subsection{The need for information systems in shared mobility}

    %Die jetzige Zeit ermöglicht durch die schnellere Entwicklung von IS die Möglichkeiten von vehicle sharing, vergleich zu früheren Papern

    Since their establishment, information technology has been a driver of innovation and productivity in organizations and around the world (CIT:NEEDED).
    Also, scientists are more and more interested in how to use information systems to create a greener, more sustainable future and tackle the challenged of climate change (\cite{Brocke2013}, p.510).
    It is also important to note, that Green IS is not only about things like energy efficiency and equipment utilization of IT, but they also contribute to designing systems, that change processes to be more efficient lean and sustainable (\cite{Brocke2013}, p.510).
    A part of this are systems that further enable the sharing economy, which would not have been possible without information systems (\cite{Puschmann2016}, p.93)

    The services especially for bike sharing already exist for some time, but only in recent years, they have gained popularity, a big part of this success can be attributed to information systems (\cite{Chen2020}, p.334).

    Because of the spontaneous nature of shared mobility services, smartphones and information systems are especially important for the success of shared mobility and shared mobility platforms (\cite{Aguilera2019}, p.19).
    Before it was not possible to geolocate cars in advance and reserve them, but with the rise of the smartphone it has been made much easier for drivers (\cite{Aguilera2019}, p.22).
    It also allows much more spontaneous rentals of cars, instead of having to wait for the taxi, the driver is immediatly able to drive and immediately fill their demand, instead of it being asynchronous.
    This is also a more efficient use of the vehicle, since it more or less directly travels to the destination instead of a taxi driver first having to get there (\cite{Aguilera2019}, p.22).
    In addition to that, the cost goes down, because of the generally more efficient distance and having to only pay the cost of the car and the platform instead of having to pay a driver to drive to the destination (\cite{Aguilera2019}, p.22).
    Lastly, this goes for ride hailing services instead of shared mobility services, the platforms grant more safety than otherwise through ratings and platforms like Uber making these ratings visible to both sides.
    This leads to the driver and the passenger being safer, because there is a system to weed out people that misbehave, this is something traditional taxis are lacking (\cite{Aguilera2019}, p.23, \cite{Canzler2016}, p.10).
    One is at the mercy of the taxi company deciding who should be allowed to drive and wo should not instead of crowdsourced information.
    This speaks to the first 2 values information systems bring to the table for shared mobility, spontaneity and security.

    For example the Oslo region saw a steep increase in the number of Oslo bike users, with progressing time.
    The system exists since 2002 and in recent years it went from 950,000 trips by 29,000 users in 2015 to 2,7 Million trips by 77,000 users in 2017 (\cite{Boecker2020}, p.392).
    The newest generation of bike sharing is the free floating kind, with 3rd, generation bike sharing, users were bound to stations, with 4th generation bike sharing stations are not necessary (\cite{Chen2020}, p.336).
    The mobile access required for this 4th generation of bike sharing has been a catalyst of its growth (\cite{Chen2020}, p.334), because it allows users to spontaneously choose bicycles in their vicinity (\cite{Aguilera2019}, p.19).

    Furthermore, IS allow to let the trips be automatically billed and monitored allowing a level of fleet control that not possible before, because too many vehicles at a given station lower the return on investment, while too few vehicles mean lost demand (\cite{Laporte2018}, p.106).
    This also translates to free floating vehicles, but instead of stations, one has to think about the density of the vehicles.
    Furthermore, the relatively new phenomenon of free floating scooters also builds on this spontaneity, and all of the known providers had no other options than to book vehicles via their app.
    Also the providers make use of apps for people that have to recharge the devices or swap the batteries, to look at which scooters to pick up where and which are allowed to take for recharging (\cite{LIME2022}).

    \subsection{Public Transit Schedules}

    The efficiency of public transit systems is a very crucial factor, when it comes to the cost effective usage of the resources, the provider has available.
    Cost effectiveness is very important for providers since public transit needs to be available for every one, as public transit has the social mission to keep the population mobile (\cite{Desaulniers2007}, p.69).
    As the fares have to be low, public transit providers have to plan very carefully with their available resources, meaning their fleet, maintenance facilities, bus and train depots and staff (\cite{Desaulniers2007}, p.69).
    These efficiency factors are all controlled by one major thing: The planning of the public transit, being strategic, tactical and operational, but also long term decisions like the design of routes and networks (\cite{Desaulniers2007}, p.69).

    One of these efficiency factors is the one with most interest for us: The operational planning, because it includes vehicle scheduling.
    Vehicle scheduling is arguably one of the most important of these planning dimensions, as it is the interface between the provider and the passenger.
    To be able to schedule vehicles, one first has to know, how many passengers will be using which route at which time (\cite{Desaulniers2007}, p.78).
    To plan that, the passenger gets seen as a node in a network, that wants to travel from one place to another for the lowest cost, generally meaning time (\cite{Desaulniers2007}, p.78).
    The solution of the problem also depends on how one calculates the cost function, generally cost means saving time, but also other factors can come into play, like how to handle overcrowding or how to handle two potential paths, that have the same time cost (\cite{Desaulniers2007}, p.79).
    Older approaches just viewed this as a simple shortest path problems, and solved it that way, more modern and accurate are stochastic approaches, since times of passenger and train/bus arrivals can vary (\cite{Desaulniers2007}, p.79).
    Although the stochastic approach makes more sense, it is a problem that gets difficult very quickly because subpaths do not necessarily concatenate, so one has to evaluate an exponential number of subtrips (\cite{Desaulniers2007}, p.79).
    A good model to put this to action is a multi agent based one, where from population data, learning humans with different preferences get simulated with them learning daily (\cite{Wahba2005}, p.102).
    However, of course the accuracy of the model is highly determinant of having accurate population data, which is of course difficult, but it shows a way how passenger demand can be simulated (\cite{Wahba2005}, p.102).

    As soon as providers have a good simulation of the passenger demand, they can try to put this demand in to public transit plans.
    The providers have 2 other major variables to look at, their number of vehicles and the number of
    %Deadhead trips

    %Light rail vs bus etc.


    \section{Method and Hypotheses}

    To answer the research question more easily, it makes sense to subdivide the research question using some hypotheses that make sense from the background we have.

    The first hypothesis is about cars, I have already shown, that car trips tend to be longer in the sharing economy.
    Because of that, car users generally have a bigger range, thus I would argue, that car users are probably more influenced by the schedules of regional trains, since they have a bigger range aswell, and cars less frequently need the connections by bus and light rail if they want to go far.

    \begin{center}
        Hypothesis 1 (H1): Cars are influenced more by regional train schedules, than other modes of transport.
    \end{center}

    Making the hypothesis the other way around, also makes sense, due to the generally shorter trips, if users are using their shared mobility ride to connect to public transit, it would make more sense if they were to use it for light rail and bus access, but due to light rail generally being fast and having a higher range I would theorize, that bike and scooter trips are used more frequently to access light rail stations.
    This makes Hypothesis 2:

    \begin{center}
        Hypothesis 2 (H2): Bikes and Scooter are influenced more by higher0
    \end{center}

    Now coming to access and egress: Due to looking at free floating vehicles, that are not station bound, the general likelyhood of a vehicle being in the near vicinity of a station, is not guaranteed, but because access trip drivers already have the vehicle, they should be more likely to park directly there.
    So I would theorize that these access trips are more likely to end near a station, than the egress trips, since the vehicle does not have to be aquired anymore, so it is easier to map the vehicle to a station, from this emerges hypothesis 3.

    \begin{center}
        Hypothesis 3 (H3): Public Transit schedules moderate access trips more strongly than egress trips.
    \end{center}

    In the following text, I will then describe the data, and the model to answer these hypotheses and the general question.

    \subsection{Data Description and Collection}

    The data of the public transit, the shared mobility usage, as well as the cumulated service areas are provided by the chair.
    The ride data is 05.11.2019-29.02.2019 with exceptions from the 06.11.2019 - 17.11.2019 and 15.01.2019 - 16.01.2019 for a total of 103 days.
    It was scraped via the API of different FFV providers, in an interval of every 5 minutes.
    Because it was scraped by reverse engineering their providers API's unfortunately there is also incomplete data for some providers.
    The final providers whose data was used for the model are Circ (Kick Scooters), Tier (Kick Scooters), KVB (Bikes), Fordpass (Bikes) and Car2Go (Cars).
    After the cleaning step about 500000 rides remain which are on average roughly 5000 on each day.
    The data set consists of nearly all data points one would find helpful for my use case when looking at such a dataset.
    It has fuel data, datetime data, location, distance as well as pricing data although the pricing data will not be used since the influences of pricing are no part of the research goal.

    The public transit data is provided in the GTFS(General Transit Feed Specification), GTFS is a standard for transportation data, which is used by thousands of public transportation providers (\cite{GTFS}).
    Generally it's comprised of a database like structure that links the different entities for example stops, routes and trips together via unique IDs.
    The GTFS data is contained in 2 data sets one ranging from the 20.10.2019 \- 14.12.2019 and the other one 14.12.2019 \- 12.12.2020.
    The GTFS data comprises about 5000 stops, 500 routes and 100000 trips, although most of these are irrelevant, because there is only trip data for Cologne.
    %Tabelle wie viel gibt es und wie viel ist relevant

    The service area data, is two GeoJSON files, the first is a union of all the service areas for Cologne.
    The second one is a set of the different GeoJSON coordinates for every single provider in Cologne.
    The union GeoJSON file will be used to evaluate all of the data together and the single data will be used to split the different modes of transport (Car, Kick Scooters and Bikes), to better understand the individual dynamics of these modes.
    Unfortunately the single GeoJSON data was not possible to get for every single provider, so some GeoJSON serviceareas for the mode of transport specific analysis had to be approximated from other similar providers.

    Furthermore, temperature and precipitation data is collected from the DWD (\url{https://dwd.de}), because it is a relevant factor for the usage of micromobility as shown by other research (\cite{Babar2017}, p.18 and \cite{Demircan2021}, p.9).
    This temperature is collected hourly from a single Cologne weather station, and consists of 3 relevant datapoints: a variable saying if there has been precipitation within the hour, a variable stating how much precipitation there was in mm per $ m ^ 2 $ and a variable, that states the average temperature within the hour.

    \begin{figure}[htbp]
        \centering
        \includegraphics[width=\textwidth,center=\textwidth]{./Figures/union map}
        \caption{Map of Cologne showing micromobility servicearea with usage of micromobility and stations} \newline
        Legend: The hexes show all the places where micromobility has been, the darker these are, the more trips they have had, and the red dots show the public transit stations, the outline shows the service area union
        \label{fig:union_sa}
    \end{figure}

    \subsection{Data Preparation}

    To achieve the goals of the developed model, the data has to be prepared accordingly, to achieve this goal, the programming language Python (\cite{Python}) is used for the data preparation and analysis.
    Most of the packages are commonly used in research projects, so do not have to be explained, but because GTFS data is difficult to handle, a package called GTFS Kit (\cite{GTFS_Kit}) is used to handle the data, and make the data easily usable for the analysis.

    Firstly I looked at a data to get a better understanding of it, some first key oddities were:

    \begin{enumerate}
        \item Some of the provided shared mobility data is really inconsistent, and has missing days for some providers
        \item GTFS supports unusual time formats like 27:13:00, that als can't be parsed with the python date library (\cite{Python}).
        \item On the 06.11.2019 the data collection stopped and only resumed on 18.11.2019, so there were many way to long "skewed" trips, also there was an outage between the 15.01.2019 - 16.01.2019.
    \end{enumerate}

    I fixed these issues by firstly removing Lime, Bird, Driveby and Drivenow from the dataset, which is very unfortunate, because they had a lot of data points which are lost because of this, but it will make it easier to compare the other days that way.
    Lime had around 10000 datapoints per day, but there's only available data till 15.12.19, also their trip volume was 5 to 10 times larger than any other kick scooter provider, depending on the day, which makes the data more questionable, and suggests errors in the data fetching.
    Bird only had datapoints in November so it made sense to remove it as well.
    For Drivenow and Driveby it was a bit more complex because Drivenow had data till the 14.01.19, larger depending on the day which is a lot longer and also had much volume, but still not the whole duration so I removed it.
    Driveby had roughly 200 rides per day in Cologne on average, but in the beginning of 2020 the ride numbers drop significantly, about tenfold, so I removed the data, since this drop probably stems from technical difficulties.

    For the GTFS problem, I always included the hours 24:00:00-28:00:00 from the last day in the first time bucket, so the numbers are accurate.
    The problem stems from the KVB counting their days from 03am to 03am the following day, this then also translates to the GTFS data, meaning that service days are not equal to normal days.
    For the last problem, all trips that are longer than 24 hours are removed, so that these trips don't get counted, since because of the scraping, it is not possible to know if this was still the same trip, which it presumably was not, because the scraping relies on matching the IDs of the vehicles every 5 minutes, if the scraping halts, the data gets very unreliable.

    Furthemore, I changed the way to look at the trips, instead of a trip with a defined start and stop, for the planned analysis, it makes more sense to look at a trip 2 seperate things: a start and a stop event.
    This generates more data, and has the advantage, that trips, that only start or end on a station hex do not have to be removed fully.
    Depending on the regression I carried out (refer to this table \ref{tab:models}) I only looked at start, end or at both events together.
    This is to be able to see differences between those two different approaches, and comparing the public transit schedules impact on FFV usage for different as researchers like \cite{Boecker2020} have found differences in start and stops in a similar case.

    After cleaning the above mentioned easy errors, the data had to be improved and needed to be brought in shape for the 12 regressions.
    Firstly, I removed further errors like round trips, or extremely short trips by removing all trips that had the same start and stop fuel for kick scooters since they have to be extremely short then.
    Then, all of the stations and start stop events were removed that are not in the service area, because they are not relevant to the analysis, this was done by calculating a point and checking if it is on the respective modes service area or in the union service area when looking at all modes.
    To reduce the overlap between stations and have more accurate data, all start and stop events, that did not occur inside of a station hex, were removed, because it is difficult to pair them with a station that way.
    Furthermore the temperature data was assigned to the timeframes and was calculated the following ways: I took the max of the precipitation variable, so it states 1 if there was any rain in any of the hourly datapoints, and I averaged the average temperature and the amount of precipitation, so for every timebucket there is also these 3 temperature variables.

    Lastly, the stop stats mentioned above are calculated for every stop and then the mean of those values per h3 is taken, this is to provide for the overlap of the close stations in a single hex, where it is very difficult to say which station the trip belongs to, so it is averaged by h3.
    The data is then grouped first by h3 and then by time to achieve a geospatial model.

    The final, polished data, consists of the rows of dependent and independent variables put together for the entire timeframe this is the final output that is then put in to the regression.
    Altough these are the final datapoints, the results are standardized according to the MinMax Scaling principle, meaning all of the values get divided by the highest value, so that they reach from 0 to 1 and are more comparable and less dependant on the used units.
    This makes the resulting model much easier to compare to other models, since they can have other maxima in their number of FFV trips or public transit trips for example.

    \subsection{Model Development}

    \subsubsection{Measuring FFV demand in Space and Time dimension}

    Similar to the models of \cite{Willing_2017_ts}, the dependent variable will be divided in spatio temporal units.
    Firstly, the data has to be sorted spatially, this is done with the help of a concept developed by Uber called h3 (\cite{h3}).
    h3 divides the world in to hexagons, and mathematically a hexagon can always contain 7 smaller hexagons, and that's called h3 resolution.
    For the project I chose the resolution h3 09, which has an area of roughly 0,1km2 with an edge length of about 0,17km, this was the resolution that made the most sense, because of the very short walking distances between public transit stops and FFVs.
    FFVs are not a time efficient way of transport, if the walking distance gets to far.
    The widely regarded walking distance is up to 500 meters, but I argue, that because Cologne having so dense stations, this walking distance only seldom needs to be that high, so h3 resolution 9 gives the most believable numbers, while minimizing the overlap between stations as you see with the ratio value of 90,97\% of Table \ref{tab:h3}.
    Now it is important to go in to detail as why the dependent variable is not only split up spatially, but also temporally.
    Because we mainly want to explain to compare public transit schedules to FFV rentals, I want to have as clear of a picture as possible as to how these relate.
    To factor out time dependent factors, which I introduce 6 timebuckets, that are 4 hours long each(0-4am, 4-8am, 8-12am, 12-4-pm, 4-8pm and 8-12pm).
    This makes our regression coefficients more dependable, because our public transit variables, then do not have to explain time dependent factors.
    In the following I will also show more of these independent variables, mostly focused on trip comfort and I will mention time dependent factors again.

    \subsubsection{Public Transit Variables}

    For every hex and every timeframe, there are four public transit variables, that says how many trips of every category took place in the timeframe.
    The different modes of transport are light rail, bus and regional trains and the number of routes.
    These variables serve as the indicator on to how much public transit was there at the time and at the place, they will be the crucial variables I will look at in connection to hypothesis approval.

    \subsubsection{Trip Comfort Factors}

    Trip comfort factors are also a significant factor found in other research for example the research by \cite{Babar2017} has shown, that cold, hot, rainy and snowy days moderate the impact on city bus utilization of introducing ride hailing in a given city.
    I added in the mean temperature as a factor, as well as precip and the sum of the amount of the precip in the given timespace.
    Because precip and amount of precip have a high correlation, due to the sum of precip amount only being above zero, if there was precip at all, meaning the precip variable is 1, I will closely monitor them to see if I have to eliminate one of the variables.

    \subsubsection{Time Dependent Factors}

    For time dependent factors the time of the days were already mentioned, meaning the 6 timebuckets I established.
    On top of that, I also established another time dependent variable based around weekdays.
    It is a binary variable called if it is a weekday or not, from other research it has been shown, that which day it is, can have a big impact on FFVS demand. (\cite{Willing_2017_im}, p.79)

    \subsubsection{Mathematics}

    in accordance to the earlier described features, I will use an OLS regression.
    This is a tried an tested method of econometric analysis.\linebreak
    $num\_ffv\_trips = \beta_{0} + \beta_{1} \times  num\_city\_train\_trips +  \beta_{2} \times  num\_regional\_train\_trips +  \beta_{3} \times  num\_bus\_trips +
    & \beta_{4} \times  num\_routes +  \beta_{5} \times  is\_weekday +  \beta_{6} \times  precip +  \beta_{7} \times  sum\_precip\_amount +  \beta_{8} \times  mean\_temp +
    & \beta_{9} \times  timebucket\_00\_04 +  \beta_{10} \times  timebucket\_04\_08 +  \beta_{11} \times  timebucket\_08\_12 +
    & \beta_{12} \times  timebucket\_12\_16 +  \beta_{13} \times  timebucket\_16\_20 +  \beta_{14} \times  timebucket\_20\_24 + \epsilon$

    This is the formula for the linear model, by letting the model train, it will predict these $ \beta_{i} $ and with the values of $ \beta_{i} $ predict the dependent variable, for every row.
    The $\epsilon$ is the error term, describing the deviation from the linear term, $\displaystyle\epsilon + \beta_0 + \sum_{i=1} ^{14} \beta_{i} \times x_{i}$ will always be excactly amount to the dependant variable.


    \section{Results}

    \subsection{Descriptive Analysis}

    Before I go in to the results of the model, I first want to describe the datapoints that the model consists of, and showcase some interesting points about them.
    I will start with the free floating vehicle datapoints and then go into detail about the GTFS feeds.

    \subsubsection{Free Floating Vehicle Descriptive Analysis}

    \begin{table}[H]
        \centering
        \begin{tabular}{lllll}
            \toprule
            \textbf{}                      & \textbf{Union} & \textbf{Bikes} & \textbf{Cars} & \textbf{Kick Scooters} \\
            \midrule
            Observations                   & 685.968        & 51.266         & 347.434       & 95.678                 \\
            Mean Trip Duration (Minutes)   & 50             & 205            & 56            & 11                     \\
            Mean Distance (Meters)         & 2.458          & 1.751          & 3.714         & 667                    \\
            Timebuckets \\
            \ \ Timebucket 00:00-04:00     & 6\%            & 8\%            & 4\%           & 10\%                   \\
            \ \ Timebucket 04:00-08:00     & 13\%           & 13\%           & 14\%          & 11\%                   \\
            \ \ Timebucket 08:00-12:00     & 18\%           & 17\%           & 18\%          & 18\%                   \\
            \ \ Timebucket 12:00-16:00     & 23\%           & 23\%           & 23\%          & 22\%                   \\
            \ \ Timebucket 16:00-20:00     & 26\%           & 29\%           & 27\%          & 24\%                   \\
            \ \ Timebucket 20:00-24:00     & 14\%           & 11\%           & 13\%          & 16\%                   \\
            Temp Mean (Celsius)            & 6,31           & 6,42           & 6,28          & 6,27                   \\
            Precip Mean                    & 46\%           & 44\%           & 48\%          & 43\%                   \\
            Sum Precip Mean (mm per $m^2$) & 0,37           & 0,29           & 0,43          & 0,33                   \\
            Weekdays                       & 71\%           & 75\%           & 72\%          & 70\%                   \\
            \bottomrule
        \end{tabular}
        \caption{Describing data from the FFV trips}
        \label{tab:descr_rides}
    \end{table}

    If you shortly look at the data (Table \ref{tab:descr_rides}), instantly some interesting patterns emerge.
    Shown are the Bikes, the Cars and the Kick Scooters and all of them combined, firstly you will notice, that they do not excactly add up.
    This comes from the fact, that different service areas are used for the union calculation and the single calculation.
    %Noch mal mit Lime laufen lassen?
    Unfortunately, there is no good service area available for the scooters, which makes the removal from the service area remove too many scooter trips.
    This is unfortunately a dilemma, because using a too big service area, would leave in many data points, that should not be left in.
    In Figure \ref{fig:scooters_sa} you can see, the relatively small service area.
    Another interesting factor is that one can see, that the rental period for bikes is the most on averge, this is an effect, that gets created by many pretty cheap daypasses.
    Many people rent bikes close to a day, but since they still can start and stop at a stop, since it is a single ride, it still makes sense to count them since, that are no errors.
    Many people just use the bike as a way to get around quickly, just like with scooters, but that touches in to the next topic: Distance.
    The distance is intuitively the longest for cars, followed by bikes, followed by e scooters.
    This finding checks out with other similar findings, for example in \cite{Demircan2021}, bikes and scooters also varied by a factor of 2.

    When looking at the timebuckets, there are some noticeable differences for example in the first timebucket, for example scooters have 10\%, bikes 8\% of their whole rides there and cars only 4\%.
    This could have multiple reasons, first reason could just be the service area, in the scooter service area (\ref{fig:scooters_sa}) are when comparatively to the whole servicearea much more nightlife districts than in the car servicearea (Figure \ref{fig:cars_sa}).
    Another reason could be, that especially in nightlife and in the central city, cars are less convenient, so they are less frequently used, compared to scooters and bikes.
    Another thing seen in the times is, that in typical weekday work starting and end times 04-08 and 16-20, cars and bikes have an edge over scooters, while on most free hours, 20-24, 00-04, scooters have an edge over cars and bikes.
    This could indicate, that scooters are more commonly used for free time travel, and sharing bikes and cars are more commonly used to get to work.
    A reason that could explain this is, that most public transit monthly subscriptions, have the bike service included and some workplaces have deals with car sharing with "service area islands".
    Some notable examples of this include Leskan Park, Axa Headquarters and Chempark Leverkusen (Figure \ref{fig:islands}).
    This theory is also underlined by the Weekday percentage of cars and bikes also being higher than for scooters, roughly 71,43\% of days are weekdays, so it is a bit more in average for bikes and cars and a bit less for scooters.

    Coming to the weather factors, the first important thing to note is, that since this all is calculated in 4 hour timebuckets, these are not the excact values at the start or stop time, but the factors in the 4 hour time window.
    When looking at the temperature, it is a reminder again, that this is all winter data, so the average temperature is low, but the temperature for the bike is about 0,15 degrees bigger, meaning that temperature could have an effect on bike rentals and even lower temperatures lead to less bike rentals.
    Interestingly, the average temperature for scooters and cars is quite similar, altough that is the case, I think it can be at least partly be explained that due to scooters being used more in the nighttime, where the temperature is lower on average, the value gets dragged down in comparison to the car.
    Still, the car has the highest value which makes sense due to it being warm and protected from the weather.
    For precip, cars are used more in rainy intervals, and have a huge margin in sum of the precip against scooters and bikes (roughly 1/3 more).
    This could imply, that general rain, does not deter passengers of micromobility as much as medium heavy to heavy rain.

    \subsubsection{GTFS Feed Descriptive Analysis}

    \begin{table}[H]
        \centering
        \begin{adjustbox}{width=1.2\textwidth,center=\textwidth}
            \begin{tabular}{ll|llll}
                \toprule
                \textbf{Data Point}                      & \textbf{Timeframe} & \textbf{union} & \textbf{bikes} & \textbf{cars} & \textbf{kick scooters} \\
                \midrule
                Number of Stops                          & 05.11.19-14.12.19  & 432            & 348            & 405           & 63                     \\
                & 15.12.19-29.02.20  & 436            & 350            & 409           & 67                     \\
                \midrule
                Number of Station Hexes                  & 05.11.19-14.12.19  & 393            & 316            & 368           & 58                     \\
                & 15.12.19-29.02.20  & 397            & 318            & 372           & 62                     \\
                \midrule
                Avg daily Number of City Train Trips     & 05.11.19-14.12.19  & 20,38          & 23,51          & 20,54         & 41,27                  \\
                & 15.12.19-29.02.20  & 20,19          & 23,37          & 20,34         & 38,81                  \\
                \midrule
                Avg daily Number of Regional Train Trips & 05.11.19-14.12.19  & 2,20           & 2,23           & 1,88          & 6,14                   \\
                & 15.12.19-29.02.20  & 2,18           & 2,21           & 1,86          & 5,77                   \\
                \midrule
                Avg daily Number of Bus Trips            & 05.11.19-14.12.19  & 24,37          & 24,29          & 24,48         & 25,86                  \\
                & 15.12.19-29.02.20  & 24,33          & 24,33          & 24,44         & 25,73                  \\
                \bottomrule
            \end{tabular}
        \end{adjustbox}
        \caption{Describing data from the GTFS feeds}
        \label{tab:descr_feed}
    \end{table}

    If you look at the table (Table \ref{tab:descr_feed}), it has to be noted first, that within the FFV trip timeframe, there are 2 distinct GTFS feeds, hence the 2 values for everything.
    The number of stops is relatively similar for union, bikes and cars, and with that the size of the service area.
    For scooters although, the service area is much smaller, and with that the available stations.
    That leads to a much higher average number of daily city train trips and regional train trips, but not for bus trips.
    This could be explained by city trains and regional trains having their many intersections in the central city, but buses are more spaced out, because their primary purpose is to connect between the trains.

    %DESCRIBE AND ANALAYZE GTFS AND TRIP DATA

    \subsection{Model Evaluation}

    \begin{table}[H]
        \begin{adjustbox}{width=1.2\textwidth,center=\textwidth}
            \begin{tabular}{lcccccc}
                & \textbf{coef} & \textbf{std err} & \textbf{t} & \textbf{P$> |$t$|$} & \textbf{[0.025} & \textbf{0.975]} \\
                \midrule
                \textbf{Intercept}                   & 0.0073***     & 0.000            & 23.314     & 0.000               & 0.007           & 0.008           \\
                \textbf{num\_city\_train\_trips}     & 0.1051***     & 0.001            & 150.323    & 0.000               & 0.104           & 0.106           \\
                \textbf{num\_regional\_train\_trips} & 0.0748***     & 0.002            & 38.442     & 0.000               & 0.071           & 0.079           \\
                \textbf{num\_bus\_trips}             & 0.0427***     & 0.001            & 39.023     & 0.000               & 0.041           & 0.045           \\
                \textbf{num\_routes}                 & -0.0369***    & 0.002            & -18.433    & 0.000               & -0.041          & -0.033          \\
                \textbf{is\_weekday}                 & -0.0050***    & 0.000            & -29.853    & 0.000               & -0.005          & -0.005          \\
                \textbf{precip}                      & -0.0011***    & 0.000            & -6.625     & 0.000               & -0.001          & -0.001          \\
                \textbf{sum\_precip\_amount}         & -0.0062***    & 0.001            & -9.507     & 0.000               & -0.007          & -0.005          \\
                \textbf{mean\_temp}                  & 0.0041***     & 0.000            & 8.498      & 0.000               & 0.003           & 0.005           \\
                \textbf{timebucket\_00\_04}          & -0.0064***    & 0.000            & -15.768    & 0.000               & -0.007          & -0.006          \\
                \textbf{timebucket\_04\_08}          & -0.0122***    & 0.000            & -31.005    & 0.000               & -0.013          & -0.011          \\
                \textbf{timebucket\_08\_12}          & -0.0061***    & 0.000            & -15.290    & 0.000               & -0.007          & -0.005          \\
                \textbf{timebucket\_12\_16}          & 0.0095***     & 0.000            & 23.113     & 0.000               & 0.009           & 0.010           \\
                \textbf{timebucket\_16\_20}          & 0.0136***     & 0.000            & 33.798     & 0.000               & 0.013           & 0.014           \\
                \textbf{timebucket\_20\_24}          & 0.0083***     & 0.000            & 20.889     & 0.000               & 0.008           & 0.009           \\
                \bottomrule
                Standard errors in parentheses, * p<.1, ** p<.05, ***p<.01
            \end{tabular}
        \end{adjustbox}
        \caption{Regressions result from the union regression with starts and stops}
        \label{tab:union_results}
    \end{table}

    I will now explain the details of the different regression analyses in context to the main research question.
    First of all it is good to say, that all regression could produce statistically significant results.
    You see the results of the biggest, the union model here in table \ref{tab:union_results}.
    The analys, will also be split in the multiple subcategories discussed before in the model part.
    Also important to note for the analyses is that, the data is minmax scaled for better comparability, this also means, that the $\beta_{i}$, meaning the coefficients can roughly be interpreted as a percentile variable.

    \subsubsection{Public Transit Variables}

    \subsection{Hypothesis Discussion}


    \section{Discussion}

    Although I conducted the research very thoroughly there are some points that remain unclear and need to be advanced.
    Firstly I will going in to the points that are influenced by the data that I used for the research.

    The GTFS data only contains plan data and no live data, which is a problem since delays and cancellations are a common occurence among public transportation (CIT:NEEDED) and it would have been especially interesting to analyze data that factors in delays and cancellations.
    With this data it would be possible to learn if people switch spontaneously from their planned public transportation trip to an unplanned FFV trip and because of no actual data and just the plan data being there it's difficult to tell if the numbers are really accurate.

    The trip data only contains winter months, and other research also shows that these comfort factors like temperature and precipitation can influence the amount of FFV usage, this effect is even stronger for uncovered vehicles. (\cite{Demircan2021}, p. 9)
    It would be thus very interesting to have summer and winter data to compare, and see if in the summer passengers are more willing to use FFV platforms.
    Furthermore, the trip data is only retrieved every 5 minutes from the different providers, which can especially for short trips make a big difference, but because only trips that are longer than 24 hours are removed from the dataset, this is not a big problem.

    Also, the positive connection between public transit and FFV demand is just a correlation, and it is very difficult to really attribute a causality to it.
    Naturally as other research has already shown, the demand for FFV is higher in the city centres and as Cologne only has one very clear city center, the demand tends to be the strongest there.
    For example the stop Neumarkt is next to one of the 10 biggest shopping streets in Germany, and also the traffic center of the communal public transit of cologne.
    This makes it of course difficult to differentiate between the traffic that is intrinsic to this shopping street and the traffic that's there because Neumarkt is a traffic center.


    \section{Conclusion}

    \pagebreak

%%%%%%%%%%%%%%%%%%%%%%%%%%%%%%%%%%%%%%%%%%%%%%%%%%%%%%%%%%%%%
%APPENDICES
%%%%%%%%%%%%%%%%%%%%%%%%%%%%%%%%%%%%%%%%%%%%%%%%%%%%%%%%%%%%%



    \appendix
    \renewcommand*{\thesection}{\Alph{section}}\textbf{}


    \section{Appendix}
% APPENDIX A

    \begin{table}[H]
        \centering
        \begin{adjustbox}{width=1.2\textwidth,center=\textwidth}
            \begin{tabular}{lllrrl}
                \toprule
                \textbf{}        & \textbf{H3 Resolution Area} & \textbf{H3 Edge Length} & \textbf{Hexes with Trips} & \textbf{Number of Hexes with Stops} & \textbf{Percentage to stops} \\
                \midrule
                H3 Resolution 10 & 0,015km^2                   & 0,06km                  & 6516                      & 430                                 & 99,54\%                      \\
                H3 Resolution 9  & 0,1km^2                     & 0,17km                  & 1201                      & 393                                 & 90,97\%                      \\
                H3 Resolution 8  & 0,7km^2                     & 0,46km                  & 216                       & 175                                 & 40,51\%                      \\
                H3 Resolution 7  & 5,2km^2                     & 1.22km                  & 51                        & 45                                  & 10,42\%                      \\
                \bottomrule
            \end{tabular}
        \end{adjustbox}
        \caption{Comparison between different h3 resolutions, total of 432 stops in the service area, calculated with the GTFS feed from 05.10-14.12}
        \label{tab:h3}
    \end{table}

    \begin{table}[H]
        \centering
        \begin{tabular}{ll}
            \toprule
            \textbf{Transport Mode} & \textbf{Trip} \\
            \midrule
            All                     & All           \\
            All                     & Start         \\
            All                     & End           \\
            Cars                    & All           \\
            Cars                    & Start         \\
            Cars                    & End           \\
            Bikes                   & All           \\
            Bikes                   & Start         \\
            Bikes                   & End           \\
            Kick Scooters           & All           \\
            Kick Scooters           & Start         \\
            Kick Scooters           & End           \\
            \bottomrule
        \end{tabular}
        \caption{Enumeration of all the models}
        \label{tab:models}
    \end{table}

    \begin{table}[H]
        \centering
        \begin{tabular}{lll}
            \toprule
            \textbf{Provider} & \textbf{Provider Servicearea used}      & \textbf{Modal Type} \\
            \midrule
            Car2Go            & DriveNow (Car2Go merged with them)      & Cars                \\
            Tier              & Tier                                    & Kick Scooters       \\
            Circ              & Tier                                    & Kick Scooters       \\
            KVB               & Call a Bike                             & Bikes               \\
            Fordpass          & Call a Bike (They were a joint-venture) & Bikes               \\
            \bottomrule
        \end{tabular}
        \caption{Service Areas used for the single Providers in the individual models}
        \label{tab:serviceareas}
    \end{table}

    \begin{table}
        \centering
        \begin{adjustbox}{width=1.2\textwidth,center=\textwidth}
            \begin{tabular}{lllllll}
                \toprule
                \textbf{}                   & \textbf{bike\_both} & \textbf{bike\_end} & \textbf{bike\_start} & \textbf{cars\_both} & \textbf{cars\_end} & \textbf{cars\_start} \\
                \midrule
                Intercept                   & 0.0036***           & 0.0036***          & 0.0027***            & 0.0169***           & 0.0171***          & 0.0162***            \\
                & (0.0004)            & (0.0004)           & (0.0004)             & (0.0005)            & (0.0006)           & (0.0006)             \\
                num\_city\_train\_trips     & 0.0482***           & 0.0405***          & 0.0406***            & 0.0974***           & 0.0965***          & 0.0913***            \\
                & (0.0008)            & (0.0007)           & (0.0008)             & (0.0012)            & (0.0013)           & (0.0013)             \\
                num\_regional\_train\_trips & 0.0189***           & 0.0153***          & 0.0167***            & 0.0742***           & 0.0759***          & 0.0671***            \\
                & (0.0021)            & (0.0020)           & (0.0022)             & (0.0034)            & (0.0036)           & (0.0035)             \\
                num\_bus\_trips             & 0.0398***           & 0.0346***          & 0.0324***            & 0.0310***           & 0.0388***          & 0.0210***            \\
                & (0.0011)            & (0.0011)           & (0.0012)             & (0.0019)            & (0.0020)           & (0.0020)             \\
                num\_routes                 & -0.0167***          & -0.0155***         & -0.0124***           & -0.0593***          & -0.0736***         & -0.0408***           \\
                & (0.0022)            & (0.0022)           & (0.0023)             & (0.0035)            & (0.0037)           & (0.0036)             \\
                is\_weekday                 & -0.0015***          & -0.0010***         & -0.0016***           & -0.0074***          & -0.0072***         & -0.0070***           \\
                & (0.0002)            & (0.0002)           & (0.0002)             & (0.0003)            & (0.0003)           & (0.0003)             \\
                precip                      & -0.0014***          & -0.0012***         & -0.0011***           & 0.0016***           & 0.0014***          & 0.0016***            \\
                & (0.0002)            & (0.0002)           & (0.0002)             & (0.0003)            & (0.0003)           & (0.0003)             \\
                sum\_precip\_amount         & -0.0079***          & -0.0064***         & -0.0069***           & 0.0040***           & 0.0045***          & 0.0032***            \\
                & (0.0007)            & (0.0007)           & (0.0007)             & (0.0012)            & (0.0012)           & (0.0012)             \\
                mean\_temp                  & 0.0059***           & 0.0048***          & 0.0052***            & 0.0015*             & 0.0018**           & 0.0011               \\
                & (0.0005)            & (0.0005)           & (0.0005)             & (0.0008)            & (0.0009)           & (0.0009)             \\
                timebucket\_00\_04          & -0.0052***          & -0.0050***         & -0.0040***           & -0.0128***          & -0.0123***         & -0.0131***           \\
                & (0.0005)            & (0.0005)           & (0.0005)             & (0.0007)            & (0.0008)           & (0.0007)             \\
                timebucket\_04\_08          & -0.0093***          & -0.0085***         & -0.0075***           & -0.0173***          & -0.0170***         & -0.0171***           \\
                & (0.0005)            & (0.0005)           & (0.0005)             & (0.0007)            & (0.0007)           & (0.0007)             \\
                timebucket\_08\_12          & -0.0077***          & -0.0076***         & -0.0054***           & -0.0008             & -0.0022***         & -0.0001              \\
                & (0.0005)            & (0.0005)           & (0.0005)             & (0.0007)            & (0.0007)           & (0.0007)             \\
                timebucket\_12\_16          & 0.0004              & -0.0008*           & 0.0013***            & 0.0288***           & 0.0256***          & 0.0291***            \\
                & (0.0005)            & (0.0005)           & (0.0005)             & (0.0007)            & (0.0008)           & (0.0008)             \\
                timebucket\_16\_20          & 0.0052***           & 0.0037***          & 0.0049***            & 0.0408***           & 0.0412***          & 0.0368***            \\
                & (0.0005)            & (0.0005)           & (0.0005)             & (0.0007)            & (0.0007)           & (0.0007)             \\
                timebucket\_20\_24          & 0.0029***           & 0.0022***          & 0.0024***            & 0.0189***           & 0.0214***          & 0.0144***            \\
                & (0.0005)            & (0.0005)           & (0.0005)             & (0.0007)            & (0.0007)           & (0.0007)             \\
                R-squared                   & 0.0877              & 0.0666             & 0.0614               & 0.1827              & 0.1629             & 0.1564               \\
                R-squared Adj.              & 0.0876              & 0.0665             & 0.0613               & 0.1826              & 0.1628             & 0.1563               \\
                \bottomrule
                Standard errors in parentheses, * p<.1, ** p<.05, ***p<.01
            \end{tabular}
        \end{adjustbox}
        \caption{Regression Summary for Bikes and Cars}
        \label{tab:r1}
    \end{table}

    \begin{table}
        \centering
        \begin{adjustbox}{width=1.2\textwidth,center=\textwidth}
            \begin{tabular}{lllllll}
                \toprule
                \textbf{}                   & \textbf{scooters\_both} & \textbf{scooters\_end} & \textbf{scooters\_start} & \textbf{union\_both} & \textbf{union\_end} & \textbf{union\_start} \\
                \midrule
                Intercept                   & 0.0073***               & 0.0079***              & 0.0074***                & 0.0073***            & 0.0074***           & 0.0076***             \\
                & (0.0010)                & (0.0011)               & (0.0010)                 & (0.0003)             & (0.0003)            & (0.0003)              \\
                num\_city\_train\_trips     & 0.0857***               & 0.0874***              & 0.0841***                & 0.1051***            & 0.1011***           & 0.1095***             \\
                & (0.0018)                & (0.0020)               & (0.0018)                 & (0.0007)             & (0.0007)            & (0.0008)              \\
                num\_regional\_train\_trips & -0.0102**               & -0.0064                & -0.0137***               & 0.0748***            & 0.0720***           & 0.0778***             \\
                & (0.0045)                & (0.0050)               & (0.0046)                 & (0.0019)             & (0.0020)            & (0.0021)              \\
                num\_bus\_trips             & 0.0142***               & 0.0138***              & 0.0146***                & 0.0427***            & 0.0456***           & 0.0396***             \\
                & (0.0021)                & (0.0023)               & (0.0022)                 & (0.0011)             & (0.0011)            & (0.0012)              \\
                num\_routes                 & 0.0088*                 & 0.0039                 & 0.0132**                 & -0.0369***           & -0.0435***          & -0.0296***            \\
                & (0.0051)                & (0.0056)               & (0.0052)                 & (0.0020)             & (0.0020)            & (0.0022)              \\
                is\_weekday                 & -0.0064***              & -0.0066***             & -0.0062***               & -0.0050***           & -0.0048***          & -0.0052***            \\
                & (0.0006)                & (0.0007)               & (0.0006)                 & (0.0002)             & (0.0002)            & (0.0002)              \\
                precip                      & -0.0032***              & -0.0034***             & -0.0030***               & -0.0011***           & -0.0011***          & -0.0011***            \\
                & (0.0006)                & (0.0006)               & (0.0006)                 & (0.0002)             & (0.0002)            & (0.0002)              \\
                sum\_precip\_amount         & -0.0110***              & -0.0109***             & -0.0112***               & -0.0062***           & -0.0057***          & -0.0067***            \\
                & (0.0023)                & (0.0025)               & (0.0023)                 & (0.0006)             & (0.0007)            & (0.0007)              \\
                mean\_temp                  & 0.0077***               & 0.0082***              & 0.0072***                & 0.0041***            & 0.0040***           & 0.0042***             \\
                & (0.0017)                & (0.0019)               & (0.0017)                 & (0.0005)             & (0.0005)            & (0.0005)              \\
                timebucket\_00\_04          & -0.0110***              & -0.0112***             & -0.0114***               & -0.0064***           & -0.0060***          & -0.0072***            \\
                & (0.0014)                & (0.0015)               & (0.0014)                 & (0.0004)             & (0.0004)            & (0.0005)              \\
                timebucket\_04\_08          & -0.0172***              & -0.0174***             & -0.0176***               & -0.0122***           & -0.0117***          & -0.0131***            \\
                & (0.0013)                & (0.0014)               & (0.0013)                 & (0.0004)             & (0.0004)            & (0.0004)              \\
                timebucket\_08\_12          & -0.0065***              & -0.0064***             & -0.0072***               & -0.0061***           & -0.0064***          & -0.0061***            \\
                & (0.0013)                & (0.0015)               & (0.0014)                 & (0.0004)             & (0.0004)            & (0.0004)              \\
                timebucket\_12\_16          & 0.0124***               & 0.0133***              & 0.0109***                & 0.0095***            & 0.0084***           & 0.0104***             \\
                & (0.0014)                & (0.0015)               & (0.0014)                 & (0.0004)             & (0.0004)            & (0.0005)              \\
                timebucket\_16\_20          & 0.0134***               & 0.0149***              & 0.0115***                & 0.0136***            & 0.0137***           & 0.0132***             \\
                & (0.0013)                & (0.0015)               & (0.0014)                 & (0.0004)             & (0.0004)            & (0.0004)              \\
                timebucket\_20\_24          & 0.0131***               & 0.0144***              & 0.0114***                & 0.0083***            & 0.0089***           & 0.0072***             \\
                & (0.0013)                & (0.0014)               & (0.0013)                 & (0.0004)             & (0.0004)            & (0.0004)              \\
                R-squared                   & 0.1967                  & 0.1769                 & 0.1846                   & 0.2400               & 0.2224              & 0.2227                \\
                R-squared Adj.              & 0.1964                  & 0.1766                 & 0.1843                   & 0.2399               & 0.2223              & 0.2227                \\
                \bottomrule
                Standard errors in parentheses, * p<.1, ** p<.05, ***p<.01
            \end{tabular}
        \end{adjustbox}
        \caption{Regression Summary for Scooters and Rest}
        \label{tab:reg2}
    \end{table}

    \begin{figure}[htbp]
        \centering
        \includegraphics[width=\textwidth,center=\textwidth]{./Figures/bikes map}
        \caption{Map of Cologne showing bike servicearea with usage of bikes and stations} \newline
        Legend: The hexes show all the places where micromobility has been, the darker these are, the more trips they have had, and the red dots show the public transit stations, the outline shows the service area union
        \label{fig:bikes_sa}
    \end{figure}
    \begin{figure}[htbp]
        \centering
        \includegraphics[width=\textwidth,center=\textwidth]{./Figures/car map}
        \caption{Map of Cologne showing car servicearea with usage of car and stations} \newline
        Legend: The hexes show all the places where micromobility has been, the darker these are, the more trips they have had, and the red dots show the public transit stations, the outline shows the service area union
        \label{fig:cars_sa}
    \end{figure}
    \pagebreak
    \begin{figure}[htbp]
        \centering
        \includegraphics[width=\textwidth,center=\textwidth]{Figures/kick scooter map}
        \caption{Map of Cologne showing kick scooter servicearea with usage of kick scooters and stations} \newline
        Legend: The hexes show all the places where micromobility has been, the darker these are, the more trips they have had, and the red dots show the public transit stations, the outline shows the service area union
        \label{fig:scooters_sa}
    \end{figure}
    \begin{figure}[htbp]
        \centering
        \includegraphics[width=\textwidth,center=\textwidth]{Figures/service area Islands}
        \caption{Service Area Islands} \newline
        Closer look at the "Service Area Islands", at the top left you see the Leverkusen Chempark, at the bottom right you can see the Axa HQ, and to the right above it, you can see the Leskan Park, this is not in the general car service area, but only in the union service area in Figure \ref{fig:union_sa}
        \label{fig:islands}
    \end{figure}
    \pagebreak

    \printbibliography[heading=bibintoc]

\end{document}
