\documentclass[12pt, onecolumn]{article}

\title{Analyzing the Effects of Public Transit Schedules on Shared Mobility Usage}
\author{Jeremy Meidinger\\[0.5cm]{Supervisor: Janik Muires\\[0.5cm]{Chair for Information Systems for Sustainable Society\\[0.5cm]{University of Cologne}}}} % Sets authors name
\date{\today} % Sets date for date compiled

\usepackage[backend=biblatex, giveninits=true, style=authoryear-icomp, maxcitenames=2, ibidtracker=false]{biblatex}
\addbibresource{main.bib}

\usepackage{hyperref}
\hypersetup{
    colorlinks,
    citecolor=black,
    filecolor=black,
    linkcolor=black,
    urlcolor=blue
}

\usepackage{titling}
\usepackage{amssymb}
\usepackage{textcomp}
\usepackage{booktabs}
\usepackage{longtable} % for 'longtable' environment
\usepackage{pdflscape} % for 'longtable' environment
\usepackage{graphicx} % for 'landscape' environment
\renewcommand\maketitlehooka{\null\mbox{}\vfill}
\renewcommand\maketitlehookd{\vfill\null}

\begin{document}


    \pagenumbering{roman}

    \begin{titlepage}
        \maketitle
        \thispagestyle{empty}
    \end{titlepage}

    \pagebreak

    \setcounter{tocdepth}{3}
    \tableofcontents
    \thispagestyle{empty}
    \pagebreak

    \pagenumbering {arabic}
    \setcounter{page}{3}


    \section{Problem Statement}
    Public transit is a very important factor in urban mobility.
    It is a very efficient way to transport many people in a short amount of time, but a lot of people do not use it, because of it's shortcomings.
    One of these shortcomings is that it is not always possible to use public transit, because of the public transit schedule.
    If one wants to travel from point A to point B, he/she has to wait for the public transit to arrive.
    Another option is to use a shared mobility system, which is essentially a combination of public transit and other modes of transport.
    These means of transport are "publicly" available to all people, if they are willing to pay the ride fares and charges the provider demands.
    This has the advantage that as long as there is a shared mobility system, people can travel freely, without having to worry about the public transit schedule.
    Shared mobility has many modes of transport some of them are: bike, car, kick scooter, and scooter.
    Though there has been research linking public transit to shared mobility, there is still a lot of research missing, most of it is focused on the last mile of public transit.
    There has been no concurrent research on the effects of public transit schedules on shared mobility usage.


    \section{Research Objective}

    The research objective is to find out if there is any link between public transit schedules and shared mobility usage, especially if a less frequent public transit schedule leads to a higher usage of shared mobility.
    To do that, we got historical data from the public transit schedules and shared mobility usage.
    To reach the objective, I divide it in the following substeps:
    \begin{enumerate}
        \item Get the data in shape to use for analysis.
        \item Analyze the data and try to find influencing factors in the public transit schedules that lead to higher/lower shared mobility usage.
        \item Set up a model that can be used to predict the usage of shared mobility based on the public transit schedules.
        \item Test the model and see if it can answer the question.
    \end{enumerate}

    \section{Theoretical Background}



    \pagebreak

    \printbibliography[heading=bibintoc]

\end{document}